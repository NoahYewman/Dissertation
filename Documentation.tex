%! Author = noahyewman
%! Date = 27/04/2020

% To be inserted wherever seems fit.


\subsection{Module\textunderscore Create.py}

To create a simple one-dimension mesh utilising Python, the Module\textunderscore Create.py program can be used.
Module\textunderscore Create.py requires no parameters to be entered onto the command line, with the script running the user through the inputs.

Module\textunderscore Create.py can be called using:
\begin{lstlisting}[style=BashInputStyle]
python3 module_create.py
\end{lstlisting}

The module parameters are:

\begin{itemize}
    \item \inlsh{coordinates}: The coordinates of which the mesh will be generated between.
    \item \inlsh{nx}: The number of nodes along the x-direction.
    \item \inlsh{ny}: The number of nodes along the y-direction.
    \item \inlsh{shape\textunderscore type}: The desired shape type of the mesh, can be set to be Quadrilateral or Triangular.
    \item \inlsh{comp\textunderscore ID}: Composite ID
    \item \inlsh{chebychev}: Set to be True/False if the user wants a boundary layer made using Chebychev nodes.
\end{itemize}

\subsection{loadCAD.py}

To generate a mesh from a .stp file, the loadCAD.py program can be used to import the CAD file and generate a high-order mesh using one command.

To generate a surface mesh of order 15, with a minimum delta of 0.04, a maximum delta of 0.2 and an epsilon value of 0.02 use the following command:

\begin{lstlisting}[style=BashInputStyle]
python3 loadCAD.py 3d_sphere.stp 3d_sphere_high_order.xml 16 False 0.04 0.2 0.02
\end{lstlisting}

The output file will be a .xml which must be converted into a .vtu using the FieldConvert utility to view the mesh.

To do this, use the command:

\begin{lstlisting}[style=BashInputStyle]
FieldConvert 3d_sphere_high_order.xml 3d_sphere_high_order.vtu
\end{lstlisting}

The parameters for this module are:

\begin{itemize}
  \item \inlsh{infile}: The name of the file to be meshed.
  \item \inlsh{outfile}: Output file name (.xml).
  \item \inlsh{nummode}: The order of the mesh.
  \item \inlsh{mindelta}: The minimum delta.
  \item \inlsh{maxdelta}: The maximum delta.
  \item \inlsh{eps}: The epsilon value.
\end{itemize}
